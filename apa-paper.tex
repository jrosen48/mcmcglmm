\documentclass[man]{apa6}

\usepackage{amssymb,amsmath}
\usepackage{ifxetex,ifluatex}
\usepackage{fixltx2e} % provides \textsubscript
\ifnum 0\ifxetex 1\fi\ifluatex 1\fi=0 % if pdftex
  \usepackage[T1]{fontenc}
  \usepackage[utf8]{inputenc}
\else % if luatex or xelatex
  \ifxetex
    \usepackage{mathspec}
    \usepackage{xltxtra,xunicode}
  \else
    \usepackage{fontspec}
  \fi
  \defaultfontfeatures{Mapping=tex-text,Scale=MatchLowercase}
  \newcommand{\euro}{€}
\fi
% use upquote if available, for straight quotes in verbatim environments
\IfFileExists{upquote.sty}{\usepackage{upquote}}{}
% use microtype if available
\IfFileExists{microtype.sty}{\usepackage{microtype}}{}

% Table formatting
\usepackage{longtable, booktabs}
\usepackage{lscape}
% \usepackage[counterclockwise]{rotating}   % Landscape page setup for large tables
\usepackage{multirow}		% Table styling
\usepackage{tabularx}		% Control Column width
\usepackage[flushleft]{threeparttable}	% Allows for three part tables with a specified notes section
\usepackage{threeparttablex}            % Lets threeparttable work with longtable

% Create new environments so endfloat can handle them
% \newenvironment{ltable}
%   {\begin{landscape}\begin{center}\begin{threeparttable}}
%   {\end{threeparttable}\end{center}\end{landscape}}

\newenvironment{lltable}
  {\begin{landscape}\begin{center}\begin{ThreePartTable}}
  {\end{ThreePartTable}\end{center}\end{landscape}}

  \usepackage{ifthen} % Only add declarations when endfloat package is loaded
  \ifthenelse{\equal{\string man}{\string man}}{%
   \DeclareDelayedFloatFlavor{ThreePartTable}{table} % Make endfloat play with longtable
   % \DeclareDelayedFloatFlavor{ltable}{table} % Make endfloat play with lscape
   \DeclareDelayedFloatFlavor{lltable}{table} % Make endfloat play with lscape & longtable
  }{}%



% The following enables adjusting longtable caption width to table width
% Solution found at http://golatex.de/longtable-mit-caption-so-breit-wie-die-tabelle-t15767.html
\makeatletter
\newcommand\LastLTentrywidth{1em}
\newlength\longtablewidth
\setlength{\longtablewidth}{1in}
\newcommand\getlongtablewidth{%
 \begingroup
  \ifcsname LT@\roman{LT@tables}\endcsname
  \global\longtablewidth=0pt
  \renewcommand\LT@entry[2]{\global\advance\longtablewidth by ##2\relax\gdef\LastLTentrywidth{##2}}%
  \@nameuse{LT@\roman{LT@tables}}%
  \fi
\endgroup}


\ifxetex
  \usepackage[setpagesize=false, % page size defined by xetex
              unicode=false, % unicode breaks when used with xetex
              xetex]{hyperref}
\else
  \usepackage[unicode=true]{hyperref}
\fi
\hypersetup{breaklinks=true,
            pdfauthor={},
            pdftitle={How engagement during out-of-school time STEM programs predicts changes in motivation in STEM},
            colorlinks=true,
            citecolor=blue,
            urlcolor=blue,
            linkcolor=black,
            pdfborder={0 0 0}}
\urlstyle{same}  % don't use monospace font for urls

\setlength{\parindent}{0pt}
%\setlength{\parskip}{0pt plus 0pt minus 0pt}

\setlength{\emergencystretch}{3em}  % prevent overfull lines


% Manuscript styling
\captionsetup{font=singlespacing,justification=justified}
\usepackage{csquotes}
\usepackage{upgreek}



\usepackage{tikz} % Variable definition to generate author note

% fix for \tightlist problem in pandoc 1.14
\providecommand{\tightlist}{%
  \setlength{\itemsep}{0pt}\setlength{\parskip}{0pt}}

% Essential manuscript parts
  \title{How engagement during out-of-school time STEM programs predicts changes
in motivation in STEM}

  \shorttitle{Engagement in STEM}


  \author{Joshua Rosenberg\textsuperscript{1}, Patrick Beymer\textsuperscript{1}, \& Jennifer Schmidt\textsuperscript{1}}

  % \def\affdep{{"", "", ""}}%
  % \def\affcity{{"", "", ""}}%

  \affiliation{
    \vspace{0.5cm}
          \textsuperscript{1} Michigan State University  }

  \authornote{
    This paper is to be presented at the 2018 Annual Meeting of the American
    Educational Research Association, New York, NY.
    
    Correspondence concerning this article should be addressed to Joshua
    Rosenberg, Postal address. E-mail:
    \href{mailto:jrosen@msu.edu}{\nolinkurl{jrosen@msu.edu}}
  }


  \abstract{Enter abstract.}
  \keywords{Engagement \\

    \indent Word count: X
  }





\usepackage{amsthm}
\newtheorem{theorem}{Theorem}
\newtheorem{lemma}{Lemma}
\theoremstyle{definition}
\newtheorem{definition}{Definition}
\newtheorem{corollary}{Corollary}
\newtheorem{proposition}{Proposition}
\theoremstyle{definition}
\newtheorem{example}{Example}
\theoremstyle{definition}
\newtheorem{exercise}{Exercise}
\theoremstyle{remark}
\newtheorem*{remark}{Remark}
\newtheorem*{solution}{Solution}
\begin{document}

\maketitle

\setcounter{secnumdepth}{0}



\section{Introduction}\label{introduction}

\section{Literature Review}\label{literature-review}

\subsection{Out-of-school time STEM
programs}\label{out-of-school-time-stem-programs}

Out-of-school time (OST) STEM programs have started focusing on STEM
content due to the lack of individuals choosing to focus on STEM
careers. Although adaptive outcomes have been found while youth are
attending OST STEM programs, little research has focused on outcomes
after the program has ended. Often, the goal of OST STEM programs is to
increase youth's long-term interest and competence in STEM. Therefore,
examining youth interest and competence in STEM after spending time in
the program is key. Framing the study around Emergent Motivation Theory
(Csikszentmihalyi, 1990), we use a profile-oriented approach to
investigate the relationship between youths' momentary engagement and
their interest and perceived competence at program completion.

\section{The Present Study}\label{the-present-study}

Research questions include: 1) What momentary profiles emerge? 2) What
profiles are predictive of interest and perceived competence after
attending an OST STEM program?

\section{Method}\label{method}

We report how we determined our sample size, all data exclusions (if
any), all manipulations, and all measures in the study.

\subsection{Participants}\label{participants}

Data were collected from 203 youth in nine OST programs, each lasting
four weeks. Through an Experience Sampling Method (ESM) approach, youth
were signaled through mobile phones, yielding 2,463 total ESM responses.
Youth were asked to complete a survey before and after the program. A
two-step cluster analysis was used to identity momentary profiles.
Multilevel Modeling was used to account for the nesting of momentary
responses within the nine OST STEM programs.

\subsection{Material}\label{material}

\subsection{Procedure}\label{procedure}

\subsection{Data analysis}\label{data-analysis}

We used R (Version 3.4.3; R Core Team, 2017) and the R-package
\emph{papaja} (Version 0.1.0.9655; Aust \& Barth, 2017) for all our
analyses.

\section{Results}\label{results}

\subsection{Maximum Likelihood
Estimation}\label{maximum-likelihood-estimation}

This model uses Maximum Likelihood (ML) estimation. Note that this model
does not account for the error in the predictions for engagement when
relating repeated measures engagement and post interest. The first
(model 0a) does not include any covariates while the second (model 0b)
adds gender, under-represented minority (URM) status, and pre-interest.

\begin{tabular}{l|l|l|r|r|r}
\hline
effect & group & term & estimate & std.error & statistic\\
\hline
fixed & fixed & (Intercept) & 1.6870529 & 0.2782234 & 6.0636625\\
\hline
fixed & fixed & rm\_engagement\_BLUP & 0.4668654 & 0.1095668 & 4.2610108\\
\hline
fixed & fixed & gender\_female & -0.0932999 & 0.1207901 & -0.7724133\\
\hline
fixed & fixed & urm & -0.2056539 & 0.1629742 & -1.2618804\\
\hline
fixed & fixed & pre\_interest & 0.5309028 & 0.0730997 & 7.2627265\\
\hline
ran\_pars & program\_ID & sd\_(Intercept) & 0.1577843 & NA & NA\\
\hline
ran\_pars & Residual & sd\_Observation & 0.6841791 & NA & NA\\
\hline
\end{tabular}

\begin{verbatim}
##               Effect   Rsq upper.CL lower.CL
## 1              Model 0.392    0.510    0.288
## 5       pre_interest 0.296    0.415    0.184
## 2 rm_engagement_BLUP 0.112    0.221    0.034
## 4                urm 0.011    0.071    0.000
## 3      gender_female 0.004    0.052    0.000
## correlation between BLUP and interest 0.334664
\end{verbatim}

\subsection{Markov Chain Monte Carlo
Estimation}\label{markov-chain-monte-carlo-estimation}

These models use Markov Chain Monte Carlo (MCMC) estimation (via the
\textbf{MCMCglmm} package). The first (model 1a) does not include any
covariates while the second (model 1b) adds gender, under-represented
minority (URM) status, and pre-interest.

\begin{verbatim}
## 
##  Iterations = 3001:12991
##  Thinning interval  = 10
##  Sample size  = 1000 
## 
##  DIC: 9872.511 
## 
##  G-structure:  ~us(trait):participant_ID
## 
##                                                      post.mean l-95% CI
## traitrm_engagement:traitrm_engagement.participant_ID    0.3375  0.26378
## traitpost_interest:traitrm_engagement.participant_ID    0.1078  0.03689
## traitrm_engagement:traitpost_interest.participant_ID    0.1078  0.03689
## traitpost_interest:traitpost_interest.participant_ID    0.6361  0.51238
##                                                      u-95% CI eff.samp
## traitrm_engagement:traitrm_engagement.participant_ID   0.4226    717.8
## traitpost_interest:traitrm_engagement.participant_ID   0.1847   1000.0
## traitrm_engagement:traitpost_interest.participant_ID   0.1847   1000.0
## traitpost_interest:traitpost_interest.participant_ID   0.7794   1000.0
## 
##                ~us(trait):program_ID
## 
##                                                  post.mean l-95% CI
## traitrm_engagement:traitrm_engagement.program_ID     5.775    1.496
## traitpost_interest:traitrm_engagement.program_ID     5.587    1.557
## traitrm_engagement:traitpost_interest.program_ID     5.587    1.557
## traitpost_interest:traitpost_interest.program_ID     5.965    1.218
##                                                  u-95% CI eff.samp
## traitrm_engagement:traitrm_engagement.program_ID    12.18     1093
## traitpost_interest:traitrm_engagement.program_ID    12.01     1000
## traitrm_engagement:traitpost_interest.program_ID    12.01     1000
## traitpost_interest:traitpost_interest.program_ID    12.81     1000
## 
##  R-structure:  ~idh(trait):units
## 
##                          post.mean l-95% CI u-95% CI eff.samp
## traitrm_engagement.units    0.4085   0.3864   0.4309     1000
## traitpost_interest.units    0.0001   0.0001   0.0001        0
## 
##  Location effects: cbind(rm_engagement, post_interest) ~ -1 + trait:gender_female + trait:urm + trait:pre_interest 
## 
##                                  post.mean l-95% CI u-95% CI eff.samp
## traitrm_engagement:gender_female  -0.05842 -0.23983  0.13108   1000.0
## traitpost_interest:gender_female  -0.20243 -0.46167  0.03909    896.5
## traitrm_engagement:urm             0.09985 -0.18320  0.35113    891.7
## traitpost_interest:urm            -0.11672 -0.45906  0.21206   1000.0
## traitrm_engagement:pre_interest    0.18050  0.06142  0.30424   1073.6
## traitpost_interest:pre_interest    0.35013  0.19166  0.51467    763.2
##                                   pMCMC    
## traitrm_engagement:gender_female  0.510    
## traitpost_interest:gender_female  0.112    
## traitrm_engagement:urm            0.434    
## traitpost_interest:urm            0.498    
## traitrm_engagement:pre_interest   0.002 ** 
## traitpost_interest:pre_interest  <0.001 ***
## ---
## Signif. codes:  0 '***' 0.001 '**' 0.01 '*' 0.05 '.' 0.1 ' ' 1
##      var1 
## 0.2410096
\end{verbatim}

\subsection{Summary of findings}\label{summary-of-findings}

To calculate these correlations, I calculated the partial R\^{}2 (i.e.,
the change in R-squared when adding the predictor) for the BLUPs for
engagement, which I think helps to compare the results with the two
approaches.

\begin{itemize}
\tightlist
\item
  BLUPs / lme4 model

  \begin{itemize}
  \tightlist
  \item
    Correlation between BLUPs for engagement and post-interest without
    covariates: .298
  \item
    Correlation between BLUPs for engagement and post-interest with
    covariates (pre-interest, gender, and URM status): .334
  \end{itemize}
\item
  MCMC / mcmcGLMM model

  \begin{itemize}
  \tightlist
  \item
    Correlation between engagement and post-interest without covariates:
    .248
  \item
    Correlation between engagement and post-interest with covariates
    (pre-interest, gender, and URM status): .280
  \end{itemize}
\end{itemize}

However, I'm not sure this is correct. In the \textbf{lme4} models, we
used the BLUPs in a separate linear model and obtained a regression
coefficient relating repeated measures engagement to post interest,
whereas in the \textbf{MCMCglmm} models we modeled the correlation of
the random effects for repeated measures engagement and interest. One
way to compare them is to use correlations instead of a linear model to
examine the relations between the predictions (BLUPs) for repeated
measures engagement and interest, but this does not allow us to account
for covariates. To address this, could a partial correlation be examined
(possibly using the \textbf{r2glmm} package or a manual approach)? Or,
is there another way to transform the beta coefficient into a partial
correlation coefficient? Or, is there a way to interpret the correlation
between repeated measures engagement and post interest in the
\textbf{MCMCglmm} models as a regression coefficient?

\section{Discussion}\label{discussion}

\newpage

\section*{References}\label{references}
\addcontentsline{toc}{section}{References}

\hypertarget{refs}{}
\hypertarget{ref-R-papaja}{}
Aust, F., \& Barth, M. (2017). \emph{papaja: Create APA manuscripts with
R Markdown}. Retrieved from \url{https://github.com/crsh/papaja}

\hypertarget{ref-R-base}{}
R Core Team. (2017). \emph{R: A language and environment for statistical
computing}. Vienna, Austria: R Foundation for Statistical Computing.
Retrieved from \url{https://www.R-project.org/}






\end{document}
